\section{Propuesta de Solución}

El proyecto consiste principalmente de dos fases: la recopilación y procesamiento de datos, y la clasificación de imágenes. El procesamiento de datos puede abarcar un amplio espectro de posibilidades, incluyendo preprocesamiento para eliminar defectos o disminuir la cantidad de datos, y la mejora de imágenes, que incluye la manipulación de los datos para aumentar el contraste, resaltar bordes, o combinar datos de varias bandas. A continuación se exploran algunos pasos a seguir.

\subsection{Recopilación de Imágenes}

La recopilación de datos principalmente consistirá de fuentes primarias de imágenes satelitales. La fuente de estos datos será el portal de datos abiertos de Copernicus, el programa de teledetección y observación terrestre de la Agencia Espacial Europea (ESA), la cual provee acceso libre y completo a cualquier persona u organización a los datos recolectados. Estas imágenes pueden ser descargadas con varios formatos, bandas espectrales, o segmentación básica como capas adicionales, con resoluciones de entre 10 y 60 metros por píxel.

% TODO https://www.copernicus.eu/en/access-data

Adicionalmente, varias fuentes secundarias se utilizarán para la verificación de los resultados producidos por el modelo, como por ejemplo mapas y diagramas geológicos.

\subsection{Preprocesamiento de Imágenes}

El objetivo del preprocesamiento es la corrección de errores y la exclusión de datos irrelevantes o demasiado degradados como para ser de uso. Se desea mejorar la utilidad de las imágenes sin alterar su esencia ni amplificar características individuales. Algunas de las técnicas a utilizarse son:

{\bf Eliminación de ruido}: ruido en una imagen satelital puede ser causado por una variedad de razones como interferencia electrónica entre componentes o mal funcionamiento de los sensores. Este ruido puede degradar una imagen e incluso ocultar información, y la eliminación de ruido tiene por objetivo disminuir el efecto del ruido sobre los datos recolectados. Las técnicas específicas dependen del tipo de ruido, es decir si es periódico (patrones de ruido que se repiten) o aleatorio, o una combinación de ambos.

% TODO source

{\bf Corrección radiométrica}: implica la correción de datos erróneos causados por la iluminación.

% TODO expand
% TODO source but different from the previous one, I found this definition to be completely different

{\bf Georreferenciación}: las imágenes satelitales son combinadas con una referencia en un sistema de coordenadas estándar, cada píxel es asociado con una coordenada. Debido a que las imágenes capturadas por el programa Copernicus ya contienen este dato, este paso se realiza al descargar las imágenes.

{\bf Subconjuntos de bandas y creación de mosaicos}: el análisis de un subconjunto de capas y la omisión de datos de poco interés ayudan a disminuir el volumen de datos para analizar, y la creación de mosaicos a partir de multiples imágenes puede servir para cubrir áreas más amplias, o analizar estructuras que no fueron capturadas en su totalidad en una imagen.
