\section{Propuesta de Solución}

El proyecto consiste principalmente de dos fases: la recopilación y procesamiento de datos, y la clasificación de
imágenes. El procesamiento de datos puede abarcar un amplio espectro de posibilidades, incluyendo preprocesamiento para
eliminar defectos o disminuir la cantidad de datos, y la mejora de imágenes, que incluye la manipulación de los datos
para aumentar el contraste, resaltar bordes, o combinar datos de varias bandas. A continuación se exploran algunos
pasos a seguir.

\subsection{Recopilación de Imágenes}

La recopilación de datos principalmente consistirá de fuentes primarias de imágenes satelitales. La fuente de estos
datos será el portal de datos abiertos de Copernicus, el programa de teledetección y observación terrestre de la
Agencia Espacial Europea (ESA), la cual provee acceso libre y completo a cualquier persona u organización a los datos
recolectados.\autocite{copernicus-access} Estas imágenes pueden ser descargadas con varios formatos, bandas
espectrales, o segmentación básica como capas adicionales, con resoluciones de entre 10 y 60 metros por píxel.

Adicionalmente, varias fuentes secundarias se utilizarán para la verificación de los resultados producidos por el
modelo, como por ejemplo mapas y diagramas geológicos.

\subsection{Preprocesamiento de Imágenes}

El objetivo del preprocesamiento es la corrección de errores y la exclusión de datos irrelevantes o demasiado
degradados como para ser de uso. Se desea mejorar la utilidad de las imágenes sin alterar su esencia ni amplificar
características individuales. Algunas de las técnicas a utilizarse son:

{\bf Eliminación de ruido}: ruido en una imagen satelital puede ser causado por una variedad de razones como
interferencia electrónica entre componentes o mal funcionamiento de los sensores. Este ruido puede degradar una imagen
e incluso ocultar información, y la eliminación de ruido tiene por objetivo disminuir el efecto del ruido sobre los
datos recolectados. Las técnicas específicas dependen del tipo de ruido, es decir si es periódico (patrones de ruido
que se repiten) o aleatorio, o una combinación de ambos. \autocite{lillesand-remote-sensing}

{\bf Corrección radiométrica}: implica la corrección de datos erróneos causados por distorsiones específicas
provenientes de efectos atmosféricos o errores de calibración que pueden causar píxeles perdidos o bandedado en la
imagen. \autocite{lillesand-remote-sensing}

{\bf Georreferenciación}: las imágenes satelitales son combinadas con una referencia en un sistema de coordenadas
estándar, cada píxel es asociado con una coordenada. Debido a que las imágenes capturadas por el programa Copernicus ya
contienen este dato, este paso se realiza al descargar las imágenes. \autocite{lillesand-remote-sensing}

{\bf Subconjuntos de bandas y creación de mosaicos}: el análisis de un subconjunto de capas y la omisión de datos de
poco interés ayudan a disminuir el volumen de datos para analizar, y la creación de mosaicos a partir de múltiples
imágenes puede servir para cubrir áreas más amplias, o analizar estructuras que no fueron capturadas en su totalidad en
una imagen. \autocite{lillesand-remote-sensing}

\subsubsection{Mejora de Imágenes}

Estos pasos se toman para aumentar la utilidad de las imágenes, resaltando ciertas características con el objetivo de
ampliar la cantidad de información que se puede interpretar a partir de las imágenes. Algunas formas de mejora de
imágenes incluyen las siguientes:

{\bf Manipulación de contraste}: consiste en resaltar diferencias entre diferentes partes de una imagen. Existen varias
técnicas de manipulación de contraste, como la umbralización de niveles de gris, la segmentación, o el estiramiento de
contraste. La primera permite definir un nivel un nivel mínimo o máximo de brillo para cada píxel, aislando ciertas
áreas. Esto permite, por ejemplo, resaltar cuerpos de agua. La segunda permite definir ciertos niveles de los píxeles
de la imagen para realizar una segmentación. La tercera consiste en ampliar el rango de valores en una imagen, que
puede hacer algunos detalles más visibles. \autocite{lillesand-remote-sensing}

{\bf Manipulación de características espaciales}: incluye filtros espaciales, filtros convolucionales y el realce de
bordes. Filtros espaciales pueden incluir un amplio rango de filtros, y se usan para aumentar o eliminar ciertas
características de una imagen, como la aumentación de nitidez o filtro de paso bajo. Filtros convolucionales también
cumplen diversas funciones, y consisten de un cálculo por cada píxel en función a sus vecinos, lo que permite realzar
cierta información local, como la detección de bordes, y permite mantener baja la cantidad de datos, combinando datos
de muchos píxeles en uno. El realce de bordes consiste en realzar el contraste entre bordes de la imagen para aumentar
su nitidez percibida. \autocite{lillesand-remote-sensing, usgs-spatial-filters}

{\bf Manipulación de múltiples imágenes}: la creación de imágenes compuestas por la combinación de datos de varias
bandas puede ser útil para realzar detalles que de otra manera no son visibles. Por ejemplo, la combinación de bandas
de color rojo y bandas de infrarrojo cercano permite evaluar la salud de vegetación. Varios índices como NDVI, SAVI o
NDSI presentan datos normalizados al rango de 0 a 1 que evalúan relaciones de este estilo entre varias bandas de una
imagen. \autocite{lillesand-remote-sensing}

\subsubsection{Clasificación}

Este es el objetivo principal del proyecto, crear un modelo capaz de reemplazar la interpretación visual por parte de
una persona entrenada por técnicas cuantitativas para automatizar la identificación de características en una escena.
Para imágenes satelitales, esto implica el análisis de múltiples bandas de la imagen, que son en efecto dimensiones
adicionales. El objetivo final es asignar etiquetas a cada píxel de una imagen de acuerdo a una lista de posibilidades.
\autocite{lillesand-remote-sensing}

La implementación de este modelo se hará en Python, que gracias a la multitud de librerías de Machine Learning y
manipulación de imágenes facilita el prototipado y desarrollo de modelos de clasificación.

\subsubsection{Entrenamiento y Validación del Modelo}

Para el entrenamiento del modelo, el conjunto de datos recolectados será particionado aproximadamente en subconjuntos
de 70\%, 20\% y 10\% para su uso en entrenamiento, validación y prueba respectivamente. Esto se hace para que la
verificación y prueba del modelo no sean influenciadas por imágenes sobre las cuales el modelo ya fue entrenado. Esta
configuración será dinámica, es decir, podrá ser ajustada dependiendo de cuál configuración demuestra mejores
resultados.

\subsubsection{Modelo y Propuesta de Solución}

Para la solución proponemos el uso de Redes Neuronales Convolucionales debido a que su eficacia en la interpretación de
imágenes ya ha sido demostrada extensamente, tanto para la clasificación binaria como para la segmentación con
múltiples etiquetas. Se utilizarán técnicas del estado del arte para explorar la clasificación y caracterización de
imágenes satelitales del territorio Paraguayo, yendo más allá de la segmentación simple ya disponible en servicios de
imágenes satelitales.

Se utilizará un conjunto de imágenes preprocesadas y mejoradas por las técnicas mencionadas en las secciones
anteriores, etiquetados correctamente para permitir un enfoque sobre el aprendizaje supervisado para la generación de
un modelo clasificador. El resultado será un modelo capaz de clasificar imágenes de diversos lugares, no solo el área
de interés de este proyecto, lo que permitirá entrenar modelos para su uso en nuevos proyectos de forma más fácil.

El área de estudio propuesta será la zona del Parque nacional Médanos del Chaco y sus alrededores, en el noroeste del
Chaco paraguayo. Esta zona ha sido estudiada por la ocurrencia de paleocauces que se extienden en dirección a la ciudad
de Mariscal Estigarribia.
