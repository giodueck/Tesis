\section{Introducción}

Imágenes satelitales o teledetección se refiere a imágenes capturadas por un sensor montado en un satélite artificial,
para extraer información. Estas imágenes contienen información multiespectro, es decir, además de la luz visible se
toman imágenes de bandas invisibles como por ejemplo la luz infrarroja. \autocite{globalforestlink-how-sat-imaging-work}

Para la captura de estas imágenes se emplean varios métodos, que se dividen en dos categorías: sensores pasivos
recolectan radiación electromagnética reflejada del sol, mientras que sensores activos emiten su propia radiación y
captan la reflexión de la tierra. Sensores activos requieren de una cantidad importante de energía para operar, pero
tienen la ventaja de operar a cualquier hora del día y la capacidad de crear imágenes en bandas que el sol no emite.
\autocite{globalforestlink-how-sat-imaging-work}

Diferentes suelos, vegetación, o humedad reflejan diferentes bandas de radiación. Estas imágenes son usadas en varias
aplicaciones, desde Sistemas de Información Geográfica y mapas a meteorología y monitoramiento de la salud de
vegetación forestal. Un índice bastante común es el Índice de Vegetación de Diferencia Normalizada, o NDVI por sus
siglas en inglés, el cual es usado para estimar la cantidad, calidad y desarrollo de la vegetación con base a la
mediación. Estos datos ya se usan en sistemas de advertencia temprana de sequías y la predicción del rendimiento de la
agricultura en los Estados Unidos a partir de los datos de la NASA. \autocite{earthdata-vegetation}

La importancia de la producción agropecuaria y agroganadera en el Paraguay también invita a considerar estas
tecnologías para el monitoreamiento de la salud de la vegetación y el uso adecuado de la tierra. Actualmente, ya se
están empleando tecnologías de teledetección y el NDVI en el sector agrícola en aplicaciones como la detección de
malezas y predicción del orden ideal de cosecha de campos cultivados. \autocite{onesoil-agricultura-paraguay}

Con aproximadamente la mitad del territorio paraguayo hacia el norte del Río Paraguay en la región semi-árida del
Chaco, tecnologías que alivien las sequías y precipitación baja son muy valiosas, tanto para la agricultura y ganadería
en las estancias chaqueñas como para centros poblacionales aislados como por ejemplo las comunidades indígenas. Estos
pueblos generalmente se caracterizan por la probreza, que se ve manifestada en una salud deteriorada producto de la
deficitaria alimentación y falta de agua potable.

Un paleocauce es un cauce por el cual antiguamente fluía agua, como por ejemplo un antiguo lecho de un río. Los
paleocauces han sido propuestos como reservorios o conductos para el flujo subterráneo de agua dulce. Se consideran de
interés principalmente los paleocauces arenosos, y estos pueden ser aprovechados para acceder al agua en áreas en las
que la distribución habitual del agua no existe o está dificultada de alguna forma. \autocite{wikipedia-paleochannel}
Con la abundancia de paleocauces en el Chaco central (ocupan un 15\% de la región), esta propuesta es una bastante
prometedora que ya ha sido considerada en investigaciones anteriores. \autocite{conacyt-sistemas-captacion-agua}

La detección de estos paleocauces se haría a partir de imágenes satelitales en una serie temporal por medio de redes
neuronales. Las redes convolucionales son una categoría de redes neuronales especializadas para el procesamiento de
imágenes. El principio básico de su funcionamiento consiste en la convolución de grupos píxeles cercanos, una operación
que permite tener en cuenta no solo el valor de cada píxel individual, sino el contexto de los mismos.
\autocite{axiv-cnn-satellite-imaging}

El resto del escrito está compuesto por las siguientes secciones: (2) una breve descripción del proyecto, (3) los
objetivos general y específicos, (4) las bases conceptuales del proyecto, (5) el estado del arte relacionado al
proyecto, (6) la metodología de la solución, (7) la importancia del presente proyecto, y finalmente (8) se presenta el
estado actual del proyecto.

