\subsection{Teledetección}

Teledetección se refiere a la captación o detección remota de alguna señal o imagen. En este contexto nos referimos
expecíficamente a imágenes captadas por medio de un sensor montado en un satélite artificial o algun vehículo aéreo
como un avión o un drón, para extraer información. Estas imágenes contienen información multiespectro, es decir, además
de la luz visible se toman imágenes de bandas invisibles como por ejemplo la luz infrarroja.
\autocite{globalforestlink-how-sat-imaging-work} A lo largo de este proyecto, el término \enquote{teledetección} se
refiere a la captación de imágenes por medio de satélites.

Para la captura de estas imágenes se emplean varios métodos, que se dividen en dos categorías: sensores pasivos
recolectan radiación electromagnética reflejada del sol, mientras que sensores activos emiten su propia radiación y
captan la reflexión de la tierra. Sensores pasivos requieren de una cantidad importante de energía para operar, pero
tienen la ventaja de operar a cualquier hora del día y la capacidad de crear imágenes en bandas que el sol no emite.
\autocite{globalforestlink-how-sat-imaging-work}

Los primeros programas de observación de la tierra por medio de satélites surgieron en los años 70 y 80. El primero fue
el programa Landsat de los Estados Unidos en 1972, y le siguieron programas similares en India, Francia y la Unión
Europea. \autocite{esa-space-year-2007}

\subsubsection{Aplicaciones}

Imágenes satelitales proveen información muy útil para todo tipo de estadísticas en áreas relacionadas con el
territorio, como por ejemplo la agricultura, silvicultura y el estudio de uso del suelo. El estudio de la agricultura a
gran escala por medio de la teledetección se realizó por primera vez entre 1974 y 1977 por medio de datos de Landsat 1,
a cargo de la NASA, la Oficina Nacional de Administración Oceánica y Atmosférica (NOAA) y el Departamento de
Agricultura de los Estados Unidos (USDA). \autocite{allen-usda-study}

Dado que las imágenes producidas generalmente cubren toda o casi toda el área de estudio, y que suelen ser
multiespectrales, lo que provee datos que fotografías ordinarias no contienen, cualquier aplicación que involucre
estudiar un área vasta puede beneficiarse de ellas. Dependiendo de la resolución, aplicaciones que involucren detalles
más finos también las pueden aprovechar, como por ejemplo su uso en aplicaciones de mapas digitales.

\subsubsection{Características de los datos}

La calidad de imágenes recolectadas por teledetección se mide de cuatro formas, estas son su resolución espacial,
espectral, radiométrica y temporal.

{\bf Resolución espacial}: el tamaño de un píxel en una imagen rasterizada. Típicamente corresponde a un área cuadrada
de entre 1 y 1000 $m^2$.

{\bf Resolución espectral}: la longitud de onda de las diferentes bandas de frecuencia capturadas, normalmente
relacionada a la cantidad de bandas de frecuencia. El sensor Hyperion en \enquote{Earth Observing-1}, por ejemplo, observa 220
bandas entre 0,4 y 2,5 $\mu m$, con una resolución espectral de 0,10--1.11 $\mu m$ por banda.
\autocite{earth-observatory-earth-observing-1}

{\bf Resolución radiométrica}: la cantidad de niveles de intensidad de radiación detectable por el sensor. Típicamente
entre 8 y 14 bits de información, correspondiente a 256 a 16384 niveles en cada banda. La cantidad de ruido en el
sensor también afecta la resolución radiométrica.

{\bf Resolución temporal}: la cantidad de sobrevuelos del avión o satélite, importante solamente cuando se realizan
series de tiempo, promedios o mosaicos, como por ejemplo en el monitoreamiento de la agricultura.

\subsubsection{Disponibilidad de recursos}

Existen varios repositorios de datos de teledetección disponibles para usos comerciales como académicos. Los programas
de observación terrestre de la NASA y de la ESA, Landsat y Copernicus respectivamente, disponibilizan recursos por
medio de portales en la internet. Para los datos de Landsat, uno de los recursos más accesibles es Google Earth Engine,
que permite el procesamiento de imágenes en línea, de forma gratuita para usos no comerciales.
\autocite{landsat-data-access} El programa Copernicus por otro lado provee un navegador de imágenes, una forma de
descargar datos con algunos filtros, y todo esto de forma gratuita tanto para fines académicos como comerciales.
\autocite{copernicus-licences} También ofrecen un espacio de trabajo en línea, similar en propósito a Google Earth
Engine. \autocite{copernicus-ds-about}
