\section{Estado del Arte}

En este capítulo se explora el estado del arte del uso de imágenes satelitales en diversas áreas y las técnicas de
análisis relevantes para este proyecto. Esta investigación tiene el fin de entender la forma en que se aplican en sus
diversos campos de aplicación y cuáles técnicas son las más eficaces en el campo a estudiarse.

\subsection{Estrategias de búsqueda}

Para la revisión de literatura se utilizaron términos referentes a [Redes Neuronales], [Teledetección], y
[Clasificación y Detección]. Se tuvieron en cuenta principalmente obras en el idioma inglés, aunque se incluyen obras
en español también. Para la búsqueda se uaron los siguientes términos:

\begin{center}
    \begin{tabular}{ l | l }
        {\bf Términos } & {\bf Sinónimos } \\
        \hline
        Neural Network & Convolutional Neural Network \\
        \hline
        Remote Sensing & Satellite Imagery \\
        \hline
        Classification & Detection \\
    \end{tabular}
\end{center}

Las cadenas de búsqueda se construyen a partir de los términos y sus sinónimos. Las cadenas con mejores resultados
fueron
[{\it Neural Network AND Remote Sensing AND Classification }],
[{\it Convolutional Neural Network AND Remote Sensing AND Classification }],
[{\it Convolutional Neural Network AND Satellite Imagery AND Classification }], y
[{\it Convolutional Neural Network AND Satellite Imagery AND Detection }]

El motor de búsqueda utilizado fue Google Scholar, poniendo mayor enfoque en resultados provinientes de bases de datos
reconocidas y establecidas como IEEE Xplore, ScienceDirect y ArXiv.

{\bf Criterios de selección} Se incluyen artículos, papers, conferencias, y otros trabajos formales debidamente
documentados. Se establecen los siguientes criterios para juzgar si un trabajo es incluído o excluído de esta
investigación:
\begin{itemize}
    \item[] {\bf Inclusión 1} Trabajos que se enfoquen en la clasificación de imágenes satelitales por medio de redes
        neuronales en el rango de publicación de 2014 a 2024.
    \item[] {\bf Inclusión 2} Trabajos que coincidan en su contenido con los términos de búsqueda.
    \item[] {\bf Inclusión 3} Trabajos cuyo contenido sea relevante para la investigación.
    \item[] {\bf Exclusión 1} Trabajos que no contengan las palabras claves o son irrelevantes para el campo de
        investigación.
    \item[] {\bf Exclusión 2} Trabajos que se centran en un término de busqueda pero no incluyen alguno de los demás.
    \item[] {\bf Exclusión 3} Trabajos con una cantidad mayoritaria de información irrelevante para el tema estudiado.
\end{itemize}

{\bf Procedimientos de selección} El proceso de selección de trabajos se basa en responder las preguntas de
investigación presentadas en la siguiente sección, con el fin de responderlas con información válida y actual.

Se limita el número de artículos incluídos a 30, y en caso de que se supere la cantidad encontrada se filtran por medio
de los siguientes criterios:

\begin{itemize}
    \item[] {\bf P.S.1.} Los trabajos deben responder la mayor cantidad de preguntas de investigación.
    \item[] {\bf P.S.2.} Los trabajos deben contar con la mayor cantidad de incidencia de términos definidos
        anteriormente.
    \item[] {\bf P.S.3.} Artículos que incluyan las palabras clasificar, interpretar, imágenes satelitales, redes
        neuronales, redes neuronales convolucionales en su resumen, conclusión.
\end{itemize}

{\bf Extracción y síntesis de datos} Para la planilla de extracción de datos de cada estudio, se guardaron título,
autores, año de publicación, resumen, palabras claves, fuente y conclusiones relacionadas a las preguntas de
investigación. En el cuadro de la sección de resultados se listan las informaciones relevantes para responder las
preguntas de investigación de este proyecto. Para determinar la inclusión de cada artículo se realizó un análisis de
los objetivos y resultados de cada trabajo, teniendo en cuenta los criterios de selección. Para realizar la síntesis de
los datos se realizó la estrategia descriptiva, que detalla y ordena las conclusiones principales de los autores de los
artículos para una mejor compresión de las ideas principales.

\subsection{Preguntas de investigación}

El objetivo principal de este estudio es determinar cual es el estado del arte en técnicas utilizadas para clasificar y
caracterizar o interpretar imágenes satelitales por medio de redes neuronales convolucionales. Con este fin en mente,
se plantean las siguientes preguntas de investigación:

\begin{enumerate}
    \item[] {\bf P.1.} ¿Qué proyectos se están llevando adelante para clasificar y caracterizar imágenes satelitales usando redes neuronales convolucionales?
    \item[] {\bf P.2.} ¿Cuáles son las ventajas y/o desventajas de la clasificación y caracterización de imágenes satelitales usando redes neuronales convolucionales en comparación con las alternativas?
\end{enumerate}

\subsection{Resultados}

% TODO
