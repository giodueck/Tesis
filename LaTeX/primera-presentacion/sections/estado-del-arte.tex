\section{Estado del Arte}

En este capítulo se explora el estado del arte del uso de imágenes satelitales en diversas áreas y las técnicas de
análisis relevantes para este proyecto. Esta investigación tiene el fin de entender la forma en que se aplican en sus
diversos campos de aplicación y cuáles técnicas son las más eficaces en el campo a estudiarse.

\subsection{Estrategias de búsqueda}

Para la revisión de literatura se utilizaron términos referentes a [Redes Neuronales], [Teledetección], y
[Clasificación y Detección]. Se tuvieron en cuenta principalmente obras en el idioma inglés, aunque se incluyen obras
en español también. Para la búsqueda se uaron los siguientes términos:

\begin{center}
    \begin{tabular}{ l | l }
        {\bf Términos } & {\bf Sinónimos } \\
        \hline
        Neural Network & Convolutional Neural Network \\
                       & Deep Learning \\
        \hline
        Remote Sensing & Satellite Imagery \\
        \hline
        Classification & Detection \\
        \hline
        Lack of data & Small data \\
    \end{tabular}
\end{center}

Las cadenas de búsqueda se construyen a partir de los términos y sus sinónimos. Las cadenas con mejores resultados
fueron
[{\it Neural Network AND Remote Sensing AND Classification }],
[{\it Convolutional Neural Network AND Remote Sensing AND Classification }],
[{\it Convolutional Neural Network AND Satellite Imagery AND Classification }],
[{\it Convolutional Neural Network AND Satellite Imagery AND Detection }], y
[{\it Deep Learning AND Remote Sensing AND Small Data }]

El motor de búsqueda utilizado fue Google Scholar, poniendo mayor enfoque en resultados provinientes de bases de datos
reconocidas y establecidas como IEEE Xplore, ScienceDirect y ArXiv.

{\bf Criterios de selección} Se incluyen artículos, papers, conferencias, y otros trabajos formales debidamente
documentados. Se establecen los siguientes criterios para juzgar si un trabajo es incluído o excluído de esta
investigación:
\begin{itemize}
    \item[] {\bf Inclusión 1} Trabajos que se enfoquen en la clasificación de imágenes satelitales por medio de redes
        neuronales en el rango de publicación de 2014 a 2024.
    \item[] {\bf Inclusión 2} Trabajos que coincidan en su contenido con los términos de búsqueda.
    \item[] {\bf Inclusión 3} Trabajos cuyo contenido sea relevante para la investigación.
    \item[] {\bf Exclusión 1} Trabajos que no contengan las palabras claves o son irrelevantes para el campo de
        investigación.
    \item[] {\bf Exclusión 2} Trabajos que se centran en un término de busqueda pero no incluyen alguno de los demás.
    \item[] {\bf Exclusión 3} Trabajos con una cantidad mayoritaria de información irrelevante para el tema estudiado.
\end{itemize}

{\bf Procedimientos de selección} El proceso de selección de trabajos se basa en responder las preguntas de
investigación presentadas en la siguiente sección, con el fin de responderlas con información válida y actual.

Se limita el número de artículos incluídos a 20, y en caso de que se supere la cantidad encontrada se filtran por medio
de los siguientes criterios:

\begin{itemize}
    \item[] {\bf P.S.1.} Los trabajos deben responder la mayor cantidad de preguntas de investigación.
    \item[] {\bf P.S.2.} Los trabajos deben contar con la mayor cantidad de incidencia de términos definidos
        anteriormente.
    \item[] {\bf P.S.3.} Artículos que incluyan las palabras clasificar, interpretar, imágenes satelitales, redes
        neuronales, redes neuronales convolucionales en su resumen, conclusión.
\end{itemize}

{\bf Extracción y síntesis de datos} Para la planilla de extracción de datos de cada estudio, se guardaron título,
autores, año de publicación, resumen, palabras claves, fuente y conclusiones relacionadas a las preguntas de
investigación. En el cuadro de la sección de resultados se listan las informaciones relevantes para responder las
preguntas de investigación de este proyecto. Para determinar la inclusión de cada artículo se realizó un análisis de
los objetivos y resultados de cada trabajo, teniendo en cuenta los criterios de selección. Para realizar la síntesis de
los datos se realizó la estrategia descriptiva, que detalla y ordena las conclusiones principales de los autores de los
artículos para una mejor compresión de las ideas principales.

\subsection{Preguntas de investigación}

El objetivo principal de este estudio es determinar cual es el estado del arte en técnicas utilizadas para clasificar y
caracterizar o interpretar imágenes satelitales por medio de redes neuronales convolucionales. Con este fin en mente,
se plantean las siguientes preguntas de investigación:

\begin{enumerate}
    \item[] {\bf P.1.} ¿Qué proyectos se están llevando adelante para clasificar y caracterizar imágenes satelitales
        usando redes neuronales convolucionales?
    \item[] {\bf P.2.} ¿Cuáles son las ventajas y/o desventajas de la clasificación y caracterización de imágenes
        satelitales usando redes neuronales convolucionales en comparación con las alternativas?
    \item[] {\bf P.3.} ¿Qué soluciones existen para abordar la falta de datos de entrenamiento para las redes
        neuronales convolucionales?
\end{enumerate}

\subsection{Resultados}

\begin{center}
    \begin{table}[h]
        \small
        \begin{tabular}{|c|c|c|c|}
            \hline
            \bf Proyecto & \bf Objetivos & \bf Métodos & \bf Observaciones \\
            \hline
            \autocite{langkvist-2016} & & & \\
            \hline
            \autocite{luengo-2016} & & & \\
            \hline
            \autocite{maggiori-2016-1} & & & \\
            \hline
            \autocite{sevo-2016} & & & \\
            \hline
            \autocite{zhong-2016} & & & \\
            \hline
            \autocite{sharma-2017} & & & \\
            \hline
            \autocite{pritt-2018} & & & \\
            \hline
            \autocite{rezaee-2018} & & & \\
            \hline
            \autocite{liu-2019} & & & \\
            \hline
            \autocite{amato-2023} & & & \\
            \hline
        \end{tabular}
        \caption{Algunos proyectos de clasificación de imágenes de teledetección por medio de CNN}
        \label{table:1}
    \end{table}
\end{center}

\begin{center}
    \begin{table}[h]
        \begin{tabular}{ |c|m{11cm}|c| }
            \hline
            \bf Tipo & \bf Característica & \bf Referencias \\
            \hline
            Ventaja & CNN patch-based (basado en pedazos) mejor que NN convencional o CNN basados en pixeles, SVN o RF
                    & \autocite{sharma-2017} \\
            \hline
            Ventaja & Clasificacion de datos multifuente por medio de CNN mejor que SVN y ELM & \autocite{xu-2017} \\
            \hline
            Ventaja & Clasificacion de uso de suelo por CNN mucho mejor que RF, especialmente para terrenos dificiles &
            \autocite{rezaee-2018} \\
            \hline
            Desventaja & CNN basado en píxeles comparable o peor que NN convencional, SVN o RF & \autocite{sharma-2017}
            \\
            \hline
            Desventaja & Entrenamiento de modelos basados en redes neuronales es más computacionalmente costoso que SVN
            o RF & \autocite{sharma-2017,xu-2017,rezaee-2018} \\
            \hline
        \end{tabular}
        \caption{Ventajas y desventajas de CNN en comparación con otras técnicas}
        \label{table:2}
    \end{table}
\end{center}

\begin{center}
    \begin{table}[ht!]
        \small
        \begin{tabular}{ |c|m{9.5cm}|c| }
            \hline
            \bf Técnica & \bf Descripción, $(+)$ Ventajas, $(-)$ Desventajas & \bf Referencias \\
            \hline
            \makecell{Transferencia \\ (Transfer, Fine-tuning)} & Uso de modelo preentrenado con un conjunto de datos
            relevante y ajustado con un conjunto de datos nuevo. $(+)$ Mejor rendimiento, menos datos de entrenamiento,
            mejor generalizabilidad. $(-)$ Riesgo de reducción de rendimiento con transferencia a dominio diferente,
            tamaño de modelo grande. & \makecell{\autocite{safonova-2023,maggiori-2016-0,castelluccio-2015} \\
            \autocite{nogueira-2017,zhong-2016,amato-2023}} \\
            \hline
            \makecell{Auto supervisado \\ (Self-supervised)} & Creación de un modelo con etiquetas creadas por el
            modelo, seguido de entrenamiento supervisado con etiquetas proveídas. $(+)$ Uso de datos no etiquetados,
            reconocimiento de patrones sin necesidad de etiquetación, mejor generalizabilidad. $(-)$ Computacionalmente
            caro, posibilidad de que el modelo deje de entrenarse con algunas técnicas. & \autocite{safonova-2023} \\
            \hline
            \makecell{Semi supervisado \\ (Semi-supervised)} & Mezcla de entrenamiento supervisado y no supervisado con
            conjuntos de datos etiquetados y no etiquetados. $(+)$ Uso de datos etiquetados y no etiquetados, mejor
            generalizabilidad. $(-)$ Computacionalmente caro, riesgo de \enquote{overfitting}, sensible a calidad de
            datos. & \autocite{safonova-2023} \\
            \hline
            Few-shot & Entrenar modelos para generalizar para nuevos problemas con unos pocos ejemplos etiquetados por
            clase. $(+)$ Enfocado al problema de pocos datos, adaptación rapida del modelo, mejor generalizabilidad.
            $(-)$ Complejidad limitada, riesgo de \enquote{overfitting}, sensible a calidad de datos. &
            \autocite{safonova-2023} \\
            \hline
            Zero-shot & Modelo Few-shot entrenado para reconocer clases que nunca ha visto antes. $(+)$ Adaptable a
            clases que no conoce, transferibilidad mejorada. $(-)$ Extremamente sensible a calidad de datos de nuevas
            instancias. & \autocite{safonova-2023} \\
            \hline
            \makecell{Débilmente supervisado \\ (Weakly-supervised)} & Desarrollo de un modelo con datos etiquetados
            parcialmente, imprecisamente o con ruido. $(+)$ Costo de etiquetamiento reducido, permite un modelo
            inexacto para permitir escalabilidad. $(-)$ Computacionalmente caro, menos exacto que entrenamiento
            (completamente) supervisado. & \autocite{safonova-2023} \\
            \hline
            Multi-task & Entrenamiento de modelo para reconocer patrones generales útiles para varias tareas. $(+)$
            Eficiencia de entrenamiento para varias tareas, mejor generalización, requerimiento reducido de datos.
            $(-)$ Alta complejidad de modelamiento, interferencia de tareas, escalabilidad limitada. &
            \autocite{safonova-2023} \\
            \hline
            Conjunto (Ensemble) & Combinación de muchos modelos individuales que aprendieron patrones de forma
            diferente para la predicción. $(+)$ Mejor generalizabilidad, robustez contra perturbación de datos e
            incertidumbre. $(-)$ Computacionalente caro, peor interpretabilidad que un modelo simple. &
            \autocite{safonova-2023,langkvist-2016,pritt-2018} \\
            \hline
            \makecell{Consciente del proceso \\ (Process-aware)} & Incorporación de regulación del entrenamiento por
            medio de procesos. $(+)$ Aprendizaje mecánico, mejor transferibilidad. $(-)$ Riesgo de pérdida de
            rendimiento si se depende de una suposición equivocada. & \autocite{safonova-2023} \\
            \hline
            \makecell{Validación cruzada \\ (Cross-validation)} & Entrenar y validar un modelo varias veces usando
            diferentes particiones de datos para el entrenamiento y la validación. $(+)$ Modelo menos sesgado, evita
            reportaje sobreoptimista de rendimiento, mejor generalizabilidad. $(-)$ Computacionalmente caro &
            \autocite{safonova-2023} \\
            \hline
        \end{tabular}
        \caption{Técnicas para abarcar el problema de pocos datos}
        \label{table:3}
    \end{table}
\end{center}
