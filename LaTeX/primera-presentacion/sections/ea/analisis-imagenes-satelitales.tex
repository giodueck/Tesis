\subsection{Análisis de imágenes satelitales}

Las formas más comunes de análisis de imágenes son la clasificación, la segmentación, la detección de cambios y las
series de tiempo. Existen muchas técnicas usadas en aplicaciones más específicas, como la predicción del rendimiento de
una plantación o la salud de la vegetación, la reducción de ruido o redes generativas, que no se aplican tan
directamente para el objetivo de este trabajo.

\subsubsection{Clasificación}

La clasificación es una tarea fundamental en el análisis de datos de teledetección, en el cual el objetivo es etiquetar
cada imagen, como por ejemplo \enquote{área urbana}, \enquote{bosque}, \enquote{agricultura}, etc. El proceso de
asignar etiquetas a imágenes se conoce como clasificación a nivel de imagen. \autocite{repo-satellite-image-dl}

Sin embargo, en algunos casos una imagen puede contener más de un tipo de uso de suelo, como por ejemplo un bosque con
un río que lo divide, o una ciudad con áreas comerciales y residenciales. En estos casos, clasificación a nivel de
imagen se vuelve más compleja e invuelve asignar múltiples etiquetas a cada imagen. Esto se puede lograr por medio de
una combinación de extracción de características y algoritmos de {\it machine learning} para identificar los diferentes
tipos de uso de suelo. \autocite{repo-satellite-image-dl}

Es importante no confundir la clasificación a nivel de imagen con la clasificación a nivel de píxel, también conocida
como segmentación semántica. Mientras que clasificación a nivel de imagen asigna una etiqueta a una imagen entera, la
segmentación semántica asigna una etiqueta a cada píxel de la imagen, lo que resulta en una representación detallada y
precisa del uso de suelo en una imagen. \autocite{cole-segmentation}

\subsubsection{Segmentación}

La segmentación consiste en dividir una imagen en segmentos o regiones semánticamente significativas. El proceso de
segmentación de imágenes asigna una etiqueta de clase a cada píxel de una imagen, transformándola de una grilla 2D de
píxeles a una grilla 2D de etiquetas. Una aplicación común es la segmentación de calles o edificios, donde el objetivo
es separar las calles y los edificios de otras características de la imagen. \autocite{repo-satellite-image-dl}

Para realizar esta tarea, modelos de una clase única son frecuentemente entrenados para detectar y diferenciar entre
calles y el ambiente, o edificios y el ambiente. Estos modelos se diseñan para reconocer características específicas
como el color, la textura y la forma que son típicas de una calle o un edificio para que puedan etiquetar los píxeles
que forman parte de estas estructuras en una imagen. \autocite{cole-segmentation}

Otras aplicaciones comunes se encuentran en la agricultura o clasificación de uso de suelo en una imagen. En este caso,
se utilizan modelos multiclase que son capaces de diferenciar entre varias clases en una imagen, como por ejemplo
bosques, áreas urbanas y tierra agrícola. Estos modelos son capaces de reconocer relaciones más complejas entre tipos
de uso de suelo, y permiten un entendimiento más integral del contenido de la imagen. \autocite{cole-segmentation}
