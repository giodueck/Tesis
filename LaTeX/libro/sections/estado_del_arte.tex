\section{Estado del Arte}

En este capítulo se explora el estado del arte del uso de imágenes satelitales en diversas áreas y las técnicas de análisis relevantes para este proyecto. Esta investigación tiene el fin de entender la forma en que se aplican en sus diversos campos de aplicación y cuáles técnicas son las más eficaces en el campo a estudiarse.

\subsection{Estrategias de búsqueda}

Para la revisión de literatura se utilizaron términos referentes a [Redes Neuronales], [Teledetección], y [Clasificación y Detección]. Se tuvieron en cuenta principalmente obras en el idioma inglés, aunque se incluyen obras en español también. Para la búsqueda se usaron los siguientes términos:

\begin{center}
    \begin{tabular}{ l | l }
        {\bf Términos } & {\bf Sinónimos } \\
        \hline
        Neural Network & Convolutional Neural Network \\
                       & Deep Learning \\
        \hline
        Remote Sensing & Satellite Imagery \\
        \hline
        Classification & Detection \\
        \hline
        Lack of data & Small data \\
    \end{tabular}
\end{center}

Las cadenas de búsqueda se construyen a partir de los términos y sus sinónimos. Las cadenas con mejores resultados fueron
[{\it Neural Network AND Remote Sensing AND Classification }],
[{\it Convolutional Neural Network AND Remote Sensing AND Classification }],
[{\it Convolutional Neural Network AND Satellite Imagery AND Classification }],
[{\it Convolutional Neural Network AND Satellite Imagery AND Detection }], y
[{\it Deep Learning AND Remote Sensing AND Small Data }]

El motor de búsqueda utilizado fue Google Scholar, poniendo mayor enfoque en resultados provenientes de bases de datos reconocidas y establecidas como IEEE Xplore, ScienceDirect y ArXiv.

También se incluyeron trabajos relevantes para la universidad y la realidad local del país, proveídos por los tutores.

{\bf Criterios de selección} Se incluyen artículos, papers, conferencias, y otros trabajos formales debidamente documentados. Se establecen los siguientes criterios para juzgar si un trabajo es incluido o excluido de esta investigación:

\vspace{-\topsep}
\begin{itemize}
    \setlength{\parskip}{0pt}
    \setlength{\itemsep}{0pt plus 1pt}
    \item[] {\bf Inclusión 1} Trabajos que se enfoquen en la clasificación de imágenes satelitales por medio de redes neuronales en el rango de publicación de 2014 a 2024.
    \item[] {\bf Inclusión 2} Trabajos que coincidan en su contenido con los términos de búsqueda.
    \item[] {\bf Inclusión 3} Trabajos cuyo contenido sea relevante para la investigación.
    \item[] {\bf Exclusión 1} Trabajos que no contengan las palabras claves o son irrelevantes para el campo de investigación.
    \item[] {\bf Exclusión 2} Trabajos que se centran en un término de búsqueda pero no incluyen alguno de los demás.
    \item[] {\bf Exclusión 3} Trabajos con una cantidad mayoritaria de información irrelevante para el tema estudiado.
\end{itemize}
\vspace{-\topsep}

{\bf Procedimientos de selección} El proceso de selección de trabajos se basa en responder las preguntas de investigación presentadas en la siguiente sección, con el fin de responderlas con información válida y actual.

Se limita el número de artículos incluidos a 20, y en caso de que se supere la cantidad encontrada se filtran por medio de los siguientes criterios:

\begin{itemize}
    \item[] {\bf P.S.1.} Los trabajos deben responder la mayor cantidad de preguntas de investigación.
    \item[] {\bf P.S.2.} Los trabajos deben contar con la mayor cantidad de incidencia de términos definidos anteriormente.
    \item[] {\bf P.S.3.} Artículos que incluyan las palabras clasificar, interpretar, imágenes satelitales, redes neuronales, redes neuronales convolucionales en su resumen, conclusión.
\end{itemize}

\subsection{Extracción y síntesis de datos}

Para la planilla de extracción de datos de cada estudio, se guardaron título, autores, año de publicación, resumen, palabras claves, fuente y conclusiones relacionadas a las preguntas de investigación. En el cuadro de la sección de resultados se listan las informaciones relevantes para responder las preguntas de investigación de este proyecto. Para determinar la inclusión de cada artículo se realizó un análisis de los objetivos y resultados de cada trabajo, teniendo en cuenta los criterios de selección. Para realizar la síntesis de los datos se realizó la estrategia descriptiva, que detalla y ordena las conclusiones principales de los autores de los artículos para una mejor compresión de las ideas principales.

\subsection{Preguntas de investigación}

El objetivo principal de este estudio es determinar cuál es el estado del arte en técnicas utilizadas para clasificar y caracterizar o interpretar imágenes satelitales por medio de redes neuronales convolucionales. Con este fin en mente, se plantean las siguientes preguntas de investigación:

\begin{enumerate}
    \item[] {\bf P.1.} ¿Qué proyectos se están llevando adelante para clasificar y caracterizar imágenes satelitales usando redes neuronales convolucionales?
    \item[] {\bf P.2.} ¿Cuáles son las ventajas y/o desventajas de la clasificación y caracterización de imágenes satelitales usando redes neuronales convolucionales en comparación con las alternativas?
    \item[] {\bf P.3.} ¿Qué soluciones existen para abordar la falta de datos de entrenamiento para las redes neuronales convolucionales?
\end{enumerate}

\subsection{Resultados}

Las respuestas a las preguntas de investigación se encuentran resumidas en los cuadros \ref{table:1}, \ref{table:2}, y \ref{table:3}.

\begin{center}
    \vspace{-\topsep}
    \begin{table}[h!]
        \small
        \hbadness=10000
        \begin{tabular}{|m{4cm}|m{5cm}|m{6.2cm}|}
            \hline
            \bf Proyecto & \bf Objetivos & \bf Métodos y Observaciones \\
            \hline
            Clasificación y Segmentación de Ortofotografía Satelital Usando Redes Neuronales Convolucionales. \autocite{langkvist-2016} \hspace{-\textwidth} & Explorar el uso de CNN para la clasificación por píxel completa, rápida y exacta de una ciudad pequeña. & Un CNN es apto para el análisis de imágenes multiespectrales corregidas ortográficamente, junto con un modelo de superficie digital de una pequeña ciudad. \\
            \hline
            Identificación y Mapeo de Paleocauces Utilizando Imágenes Satelitales de Alta Resolución en la Llanura Costera de la Bahía Samborombón, Argentina \autocite{luengo-2016} & Reconocimiento y análisis de paleocauces por medio de sensores remotos de los paleocauces de los ríos Samborombón y Salado, en la zona de su desembocadura, para la reconstrucción paleoambiental. & Se utilizan imágenes satelitales multiespectrales de alta resolución en combinación con algunos filtros de dirección convolucionales. Sin embargo, no se utilizan redes neuronales. \\
            \hline
            Redes Neuronales Completamente Convolucionales para Clasificación de Imágenes Satelitales \autocite{maggiori-2016-1} & Demostrar un modelo CNN que utiliza solamente capas convolucionales, sin capas tradicionales, para la clasificación de imágenes satelitales. & El modelo resultante tiene mejor rendimiento que modelos CNN con capas tradicionales y menor tiempo de entrenamiento. \\
            \hline
            Detección Automática de Objetos en Imágenes Aéreas basada en CNN \autocite{sevo-2016} & Aprovechamiento de la cantidad de imágenes de teledetección por medio de CNN para la detección automática de objetos en diversas aplicaciones. & CNNs son útiles para la detección de objetos, mejor que métodos basados en \enquote{features} y varios otros modelos CNN. \\
            \hline
            SatCNN: clasificación de datos de imágenes satelitales usando CNN ágiles \autocite{zhong-2016} & Diseño de una arquitectura CNN especializada para el análisis de imágenes satelitales en vez de adaptar un modelo existente creado para la clasificación de escenas naturales. & Pruebas en varias muestras de datos muestran una efectividad de más de 99.5\%. \\
            \hline
            Un CNN basado en pedazos (patch-based) para la clasificación de datos de teledetección \autocite{sharma-2017} & Creación de un modelo para la clasificación de imágenes de resolución media, donde estructuras finas no existen para la clasificación por píxel. & Mejora de hasta 25\% por encima de CNN basados en clasificación por píxel. \\
            \hline
            Clasificación de Imágenes Satelitales con Deep Learning \autocite{pritt-2017} & Creación de modelos CNN para la automatización del análisis de grandes cantidades de imágenes satelitales de alta resolución. & El modelo creado gana el segundo puesto en la competición de Functional Map of the World (fMoW) TopCoder, 15 de 63 clases clasificadas con exactitud de 95\%. \\
            \hline
            CNN para la Clasificación de Humedales Complejos Usando Imágenes de Teledetección Óptica \autocite{rezaee-2018} & Entrenamiento de un modelo preentrenado para la clasificación de imágenes de alta resolución. & Rendimiento mejor que RF, inclusive con menos características de entrada, con una exactitud de 94\%. \\
            \hline
            DeepSat V2: CNN de Características Aumentadas para la Clasificación de Imágenes Satelitales \autocite{liu-2019} & Creación de conjuntos de datos de imágenes satelitales etiquetadas y un modelo CNN con varias técnicas nuevas para su análisis. & El modelo resultante logra clasificar correctamente con más de 99\% de exactitud en los nuevos conjuntos de datos. \\
            \hline
            Un CNN para la Detección de Anomalías Térmicas Volcánicas en Imágenes Satelitales \autocite{amato-2023} & Entrenamiento de un modelo CNN para analizar imágenes de actividad volcánica en infrarrojo. & Un CNN preentrenado adaptado a nuevos datos seleccionados a mano logra buena clasificación. \\
            \hline
        \end{tabular}
        \caption{Algunos proyectos de clasificación y segmentación de imágenes de teledetección y CNN}
        \label{table:1}
    \end{table}
    \vspace{-\topsep}
\end{center}

\begin{center}
    \vspace{-\topsep}
    \begin{table}[h!]
        \begin{tabular}{ |c|m{11cm}|c| }
            \hline
            \bf Tipo & \bf Característica & \bf Referencias \\
            \hline
            Ventaja & CNN patch-based (basado en pedazos) mejor que NN convencional o CNN basados en píxeles, SVN o RF & \autocite{sharma-2017} \\
            \hline
            Ventaja & Clasificación de datos multifuente por medio de CNN mejor que SVN y ELM & \autocite{xu-2017} \\
            \hline
            Ventaja & Clasificación de uso de suelo por CNN mucho mejor que RF, especialmente para terrenos difíciles & \autocite{rezaee-2018} \\
            \hline
            Desventaja & CNN basado en píxeles comparable o peor que NN convencional, SVN o RF & \autocite{sharma-2017} \\
            \hline
            Desventaja & Entrenamiento de modelos basados en redes neuronales es más computacionalmente costoso que SVN o RF & \autocite{sharma-2017,xu-2017,rezaee-2018} \\
            \hline
        \end{tabular}
        \caption{Ventajas y desventajas de CNN en comparación con otras técnicas}
        \label{table:2}
    \end{table}
    \vspace{-\topsep}
\end{center}
\todo[inline]{agregar métricas al cuadro comparativo}

\begin{center}
    \vspace{-\topsep}
    \begin{table}[h!]
        \footnotesize
        \begin{tabular}{ |c|m{9.5cm}|c| }
            \hline
            \bf Técnica & \bf Descripción, $(+)$ Ventajas, $(-)$ Desventajas & \bf Referencias \\
            \hline
            \makecell{Transferencia \\ (Transfer, Fine-tuning)} & Uso de modelo preentrenado con un conjunto de datos relevante y ajustado con un conjunto de datos nuevo. $(+)$ Mejor rendimiento, menos datos de entrenamiento, mejor generalizabilidad. $(-)$ Riesgo de reducción de rendimiento con transferencia a dominio diferente, tamaño de modelo grande. & \makecell{\autocite{safonova-2023,maggiori-2016-0,castelluccio-2015} \\ \autocite{nogueira-2017,zhong-2016,amato-2023}} \\
            \hline
            \makecell{Auto supervisado \\ (Self-supervised)} & Creación de un modelo con etiquetas creadas por el modelo, seguido de entrenamiento supervisado con etiquetas proveídas. $(+)$ Uso de datos no etiquetados, reconocimiento de patrones sin necesidad de etiquetación, mejor generalizabilidad. $(-)$ Computacionalmente caro, posibilidad de que el modelo deje de entrenarse con algunas técnicas. & \autocite{safonova-2023} \\
            \hline
            \makecell{Semi supervisado \\ (Semi-supervised)} & Mezcla de entrenamiento supervisado y no supervisado con conjuntos de datos etiquetados y no etiquetados. $(+)$ Uso de datos etiquetados y no etiquetados, mejor generalizabilidad. $(-)$ Computacionalmente caro, riesgo de \enquote{overfitting}, sensible a calidad de datos. & \autocite{safonova-2023} \\
            \hline
            \makecell{Débilmente supervisado \\ (Weakly-supervised)} & Desarrollo de un modelo con datos etiquetados parcialmente, de manera imprecisa o con ruido. $(+)$ Costo de etiquetamiento reducido, permite un modelo inexacto para permitir escalabilidad. $(-)$ Computacionalmente caro, menos exacto que entrenamiento (completamente) supervisado. & \autocite{safonova-2023} \\
            \hline
            % Few-shot & Entrenar modelos para generalizar para nuevos problemas con unos pocos ejemplos etiquetados por clase. $(+)$ Enfocado al problema de pocos datos, adaptación rapida del modelo, mejor generalizabilidad. $(-)$ Complejidad limitada, riesgo de \enquote{overfitting}, sensible a calidad de datos. & \autocite{safonova-2023} \\
            % \hline
            % Zero-shot & Modelo Few-shot entrenado para reconocer clases que nunca ha visto antes. $(+)$ Adaptable a clases que no conoce, transferibilidad mejorada. $(-)$ Extremamente sensible a calidad de datos de nuevas instancias. & \autocite{safonova-2023} \\
            % \hline
            % Multi-task & Entrenamiento de modelo para reconocer patrones generales útiles para varias tareas. $(+)$ Eficiencia de entrenamiento para varias tareas, mejor generalización, requerimiento reducido de datos. $(-)$ Alta complejidad de modelamiento, interferencia de tareas, escalabilidad limitada. & \autocite{safonova-2023} \\
            % \hline
            Conjunto (Ensemble) & Combinación de muchos modelos individuales que aprendieron patrones de forma diferente para la predicción. $(+)$ Mejor generalizabilidad, robustez contra perturbación de datos e incertidumbre. $(-)$ Computacionalente caro, peor interpretabilidad que un modelo simple. & \autocite{safonova-2023,langkvist-2016,pritt-2017} \\
            \hline
            % \makecell{Consciente del proceso \\ (Process-aware)} & Incorporación de regulación del entrenamiento por medio de procesos. $(+)$ Aprendizaje mecánico, mejor transferibilidad. $(-)$ Riesgo de pérdida de rendimiento si se depende de una suposición equivocada. & \autocite{safonova-2023} \\
            % \hline
            \makecell{Validación cruzada \\ (Cross-validation)} & Entrenar y validar un modelo varias veces usando diferentes particiones de datos para el entrenamiento y la validación. $(+)$ Modelo menos sesgado, evita reportaje sobre-optimista de rendimiento, mejor generalizabilidad. $(-)$ Computacionalmente caro & \autocite{safonova-2023} \\
            \hline
        \end{tabular}
        \caption{Técnicas para abarcar el problema de pocos datos}
        \label{table:3}
    \end{table}
    \vspace{-\topsep}
\end{center}

\subsubsection{Discusión de los resultados}

El campo de análisis de datos de teledetección ha visto muchos proyectos en los últimos años, y las redes convolucionales han sido una pieza crucial para crear modelos mas poderosos y eficientes que las redes neuronales convencionales. Las exploración de alternativas a redes neuronales, como RF y SVN, indica que la CNN es la tecnología más prometedora para la clasificación de una gran cantidad de datos.

La posibilidad de usar datos sin etiquetamiento en el entrenamiento indica que la falta de datos etiquetados no es razón para dejar de considerar el entrenamiento de modelos independientes. Sin embargo, gracias a la cantidad de cómputo necesaria para la tarea, el exploramiento que se realice en este trabajo necesariamente deberá ser limitado. Aun así, gracias a estos factores, el resultado debería reflejar con bastante precisión los métodos necesarios para crear modelos con objetivos poco comunes en terreno nuevo para la teledetección a nivel nacional.

También de importancia es la naturaleza altamente dinámica de los datos. Cada año la cantidad de datos de alta resolución se multiplica y se vuelve más accesible, y para mantenerse al tanto de estos cambios, son necesarios el mantenimiento y la actualización periódica de estos modelos. Por otro lado, una de las técnicas más utilizadas en los trabajos mencionados es la transferencia de modelos preentrenados por medio de \enquote{fine-tuning}, lo que hace el mantenimiento un obstáculo más fácil de superar.

