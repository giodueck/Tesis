\section{Abstract}

Satellite imaging has created a wealth of easily accessible data about the surface of the earth, but interpretation and analysis requires expertise and large amounts of time. Convolutional Neural Networks (CNN) offer a way to do this analysis in a fast and general manner, especially for segmentation and land use classification workloads.

This project focuses on the development of tools and methodologies that apply CNN architectures to the problem of large scale analysis of freely available satellite images, with the specific goal of detecting palaeochannels, a length of river or stream channel which has been abandoned, in the Chaco region in Paraguay. These geological formations are of great interest for their use as a source of groundwater where no other easily accessible source is available.

Previous research has shown that CNNs are fit for the purpose of image segmentation and object detection in satellite imagery. This project aims to show how to apply those technologies in a simple and extensible manner to this goal, and more broadly to any land use classification problem. For this purpose, a tool was developed to simplify iteration and testing of different configurations of datasets and models, called \enquote{Torchbearer}.

The study area is the Médanos del Chaco National Park and the Alluvial Fan of the Pilcomayo, which were selected based on the existence of previous surveys of palaeochannels and the ever-present interest in a safe and constant supply of water in the semi-arid Chaco region.

When used with images containing the coordinates of previous geological surveys in the Paraguayan and Argentine Chaco regions, models trained in the course of this project achieved 80-100\% accuracy.

{\bf Keywords}: CNN, Convolutional Neural Network, Earth Observation, Satellite images, Segmentation model, Land use, Palaeochannel, Chaco
