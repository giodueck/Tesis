\section{Resumen}

Las imágenes satelitales han creado una gran cantidad de datos fácilmente accesibles sobre la superficie terrestre, pero su análisis requiere de habilidad experta y una gran cantidad de tiempo. Las Redes Neuronales Convolucionales (CNN) brindan una forma de realizar este análisis de una manera rápida y generalizada, en especial para trabajos de segmentación y clasificación de uso de suelo.

Este proyecto se enfoca en el desarrollo de herramientas y metodologías que apliquen arquitecturas CNN al problema del análisis a gran escala de imágenes satelitales libremente disponibles, con el objetivo específico de la detección de paleocauces, cauces de ríos o arroyos abandonados, en la región del Chaco en Paraguay. Estas formaciones geológicas so de gran interés por su utilidad como fuente de agua subterránea en lugares en los que no se disponen de fuentes más accesibles.

Estudios anteriores han demostrado que las CNN son aptas para la segmentación y detección de objetos en imágenes satelitales. Este proyecto propone una forma de aplicar estas tecnologías de una manera simple y extensible a este problema, y más generalmente a cualquier problema de clasificación de uso de suelo. Para este propósito, se desarrolló una herramienta que simplifica la iteración y la prueba de diferentes configuraciones de conjuntos de datos y modelos, llamada \enquote{Torchbearer}.

El área de estudio son el Parque Nacional Médanos del Chaco y el Abanico Aluvial del Pilcomayo, que fueron elegidos por la existencia de estudios previos de paleocauces y por el constante interés en una fuente segura y constante de agua en la región semi-árida del Chaco.

Cuando se utilizan con imágenes que contienen las coordenadas de estudios geológicos anteriores en las regiones del Chaco Paraguayo y Chaco Argentino, modelos entrenados en el marco del proyecto lograron una precisión de 80-100\%.

{\bf Palabras clave}: CNN, Red Neuronal Convolucional, Observación Terrestre, Imágenes Satelitales, Modelo de segmentación, Uso de suelo, Paleocauce, Chaco
