\section{Discusión de los Resultados}

% Conclusion:
% - fcn models not appropriate for this task, farseg shows promise and probably needs better training data
% - unet very good in relation to the quality of the training data
% - since training data is best-effort, results are not perfect or even the best possible with the architecture
% - results useful for a first review of an area
% Strengths:
% - short training time
% - simple architectures
% - simple training data setup
% - good for active/humid palaeochannels
% - decent for colmatado palaeochannels
% Limitations:
% - noisy in urban/cultivated areas
% - very dependent on time of year: rainy seasons and droughts both have very negative impacts
% - tipically a result model good for active pc is bad for colmatado, and vice-versa
% Possible improvements:
% - 192 has nicer results than 128, larger could be even better
% - 60m images were used here, larger images could make 20m or 10m resolution useful as well
% - using professional cards or higher end cosumer hardware will lift some limitations related to memory and video-memory
% - training masks can still be improved, will likely eliminate some of the noisy output and make results more consistent
% - u-nets have variants, some are sure to improve results, e.g. unet++

En esta sección se presentan las conclusiones obtenidas a partir de los resultados de los experimentos realizados a lo largo de este trabajo Proyecto Final de Carrera. También se discuten recomendaciones para trabajos futuros así como para posibles implementaciones, se detallan las contribuciones del proyecto, y posibilidades para expandir sobre este proyecto en trabajos posteriores.

\subsection{Conclusión}

En el curso de este trabajo se han buscado e implementado métodos de creación de modelos de redes neuronales convolucionales (CNN) para su uso en el análisis de imágenes satelitales. Se consideraron tareas de clasificación y segmentación de imágenes, llegando a la conclusión que la clasificación es inadecuada para la tarea específica de la detección de paleocauces. Se evaluó la viabilidad del uso de varias arquitecturas de modelos CNN y se experimentó con los parámetros de los mismos.

Las imágenes utilizadas cubren rangos temporales a lo largo del año, teniendo en cuenta temporadas de lluvia así como sequías históricas para la selección de imágenes para el entrenamiento y la comprobación de modelos de segmentación. Se encontró un efecto detrimental atribuible a imágenes tomadas dentro de poco tiempo siguiendo precipitaciones. De manera similar, sequías prolongadas también demostraron ser de poco uso para el entrenamiento. Además de filtrar imágenes con demasiada humedad o sequía, el único criterio para determinar si una imagen es apta para el uso con modelos de segmentación fue la cobertura de nubes en la imagen. Ya que el par de satélites Sentinel tienen un tiempo de revisita de alrededor de cinco días, hay una abundancia de imágenes disponibles a escoger.

Por medio de una herramienta desarrollada para este trabajo, Torchbearer, el entrenamiento de modelos se realizó de forma paramétrica, con los modelos resultantes siendo empleados directamente en la creación de mapas de inferencia.

Entre las arquitecturas presentados en el capítulo de Experimentos y Resultados, la menos apta para la tarea de segmentación es el FCN, o red enteramente contectada. Esta arquitectura produce modelos que producen predicciones completamente uniformes.

La arquitectura FarSeg, o Foreground-Aware Relation Network, produjo resultados más prometedores. Mientras que no son predicciones muy significativas, los resultados dependen enteramente de la selección de tres bandas de la imagen. Es posible que una selección diferente a las utilizadas en los experimentos produzca resultados más concretos. Aun así, los mejores resultados de estos modelos produjeron predicciones muy poco extensas.

Una limitación potencial para estas arquitecturas son los datos de entrenamiento, creados en base a un mejor esfuerzo para recrear mapas de paleocauces ya conocidos en el Gran Chaco. Datos de mejor calidad podrían ser lo que hace viable a ambas arquitecturas.

La mejor arquitectura estudiada es U-Net, bien conocida por su capacidad de segmentación de imágenes médicas. Los resultados de estos modelos dependen también de los datos de entrenamiento, aunque parecen producir predicciones más generales. Dado que las imágenes procesadas por esta arquitectura pasan por filtros que reducen la resolución de la imagen, y luego por filtros que sintetizan datos nuevos al aumentar nuevamente la resolución, estos modelos pueden aprender los patrones deseados de forma muy general, incluso con datos de entrenamiento imperfectos.

En comparaciones hechas entre trabajos de campo en el Chaco Argentino, algunas versiones de los modelos U-Net lograron una predicción correcta en todas las áreas de sondeos geoeléctricos y pozos. Esta zona presenta muchos paleocauces que se formaron en paralelo y pueden caracterizarse como paleocauces activos, ya que presentan mucha humedad en la superficie y una gran abundancia de lagunas, ambas características clásicas de este tipo de paleocauces. Específicamente, las versiones 7, 8, 11 y 12 producen predicciones correctas en el 100\% de los sitios de sondeos o pozos, mientras que las versiones 9 y 10 producen predicciones significativas correctas en solamente 33\% de los mismos sitios, sugeriendo que su arquitectura es enteramente inadecuada.

En el Chaco Paraguayo, los modelos predicen correctamente entre ocho y nueve de las localidades de pozos dentro y fuera de paleocauces, de un total de diez. Esta zona está dotada principalmente de paleocauces colmatados, presentando una superficie arenosa, con poca humedad y poca vegetación, que conforman el Sistema Acuífero Paleocauce. En específico, las versiones 7 y 8 tienen predicciones 90\% correctas, y las versiones 11 y 12 son correctas en el 80\% de los sitios. Sin embargo, las versiones 9 y 10 nuevamente produjeron mapas muy esparcidos, con predicciones correctas menores al 30\%.

Aunque los resultados son en general mejores en zonas de paleocauces activos, con resultados entre 10\% y 20\% mejores gracias a sus distintivas características superficiales, los paleocauces colmatados son de mayor interés. Esto se debe a que paleocauces activos están dotados de lagunas y humedales, y en las zonas mencionadas, se presentan en cercanía de rios, mientras que los paleocauces colmatados del Chaco Central se presentan en un ambiente semi-árido en el cual el agua se considera un recurso mucho más escaso, especialmente en sequías cuando las reservas de agua más accesibles pueden agotarse rápidamente.

\subsection{Limitaciones}

La principal limitación en trabajos de clasificación y segmentación son los datos de entrenamiento. Si estos datos son de mala calidad, los modelos entrenados no alcanzan zu máximo potencial. Mientras que los resultados de este Proyecto Final de Carrera son muy interesantes y prometedores, esta limitación está presente. Las máscaras objetivo fueron creadas a base de un mejor esfuerzo, basadas en imágenes de baja resolución y reconocimiento visual. Mapas de mayor fidelidad pueden generar modelos que producen resultados menos ruidosos, con mejor exactitud y bordes más definidos.

Otra limitación importante fue la disponibilidad de recursos computacionales. Los recursos utilizados para este trabajo son hardware de consumidor, no siendo especializados para cargas de trabajo de entrenamiento de redes neuronales o inferencia. Este factor limita la velocidad del entrenamiento, pero impone una barrera en cuanto al tamaño y complejidad de las arquitecturas CNN utilizadas. El uso de equipamiento profesional, que cuentan con una mayor cantidad de memoria y poder de procesamiento, puede tanto acelerar el entrenamiento como permitir la implementación de modelos mucho más complejos.

En cuanto a los modelos entrenados, una limitación es la precisión de predicciones en zonas de paleocauces colmatados, y más allá, la cantidad de predicciones confundidas por zonas urbanas, calles y caminos, y campos cultivados. Esta limitación es evidente en las imágenes de la figura \ref{fig:mariscal}. Una mejora en las limitaciones discutidas anteriormente puede mitigar esta tercera.

Otra consecuencia de las dos primeras limitaciones es la restricción a imágenes de resolución baja. Debido al gran tamaño de las estructuras de paleocauces, es necesario incluir un gran contexto. En este estudio se utilizaron regiones cuadradas de 128 y 192 píxeles por la limitación de memoria disponible para el entrenamiento. Esto resulta en regiones de $7\,680m$ y $11\,520m$ con imágenes de 60 metros de resolución, y se demostró que los resultados mejoraron con la región mayor. Para utilizar imágenes de 20 o 10 metros, sería necesario no sólo aumentar la capacidad de cómputo, pero también la calidad y resolución de los datos de entrenamiento.

\subsection{Recomendaciones}

Para continuar esta línea de investigación, recomendamos que trabajos futuros se enfoquen principalmente en el problema de la recolección y preparación de datos de entrenamiento. Esto puede hacerse por medio de una especialización sobre los paleocauces colmatados del Chaco Paraguayo, o más bien mediante la creación de mapas de mayor calidad por medio de estudios geológicos.

La creación de modelos especializados podría necesitar de más datos de entrenamiento, tomadas en el área que corresponde para cada tipo, pero resultando en predicciones mucho más eficaces. El Chaco Paraguayo, que presenta un grande sistema de paleocauces colmatados, y el Chaco Argentino, con grandes cantidades de paleocauces activos, serían las posibles fuentes principales datos de entrenamiento para cada tipo de paleocauce.

Otra posibilidad es el uso de hardware profesional para permitir el análisis de ventanas más grandes y resoluciones más altas que 60 metros. Dado que los resultados con ventanas de 192 píxeles de lado son mejores que con ventanas de 128, es posible que ventanas aun más amplias produzcan predicciones aun mejores. También se pueden emplear arquitecturas más complejas, no utilizadas en este trabajo debido a esta limitación.

\subsection{Contribuciones}

Partiendo de las conclusiones en las secciones anteriores, podemos afirmar que alcanzamos los objetivos propuestos para este trabajo PFC. Las contribuciones principales del trabajo son las siguientes:

\begin{itemize}
    \item El desarrollo de metodologías y herramientas para la creación de modelos de clasificación de uso de suelo a partir de tecnologías bien establecidas en el estado del arte.
    \item La evaluación de varias arquitecturas de modelos CNN para su uso en tareas de segmentación.
\end{itemize}

\subsection{Trabajos futuros}

Este trabajo expande el estado del arte en lo referente al uso de modelos CNN en aplicaciones de segmentación y determinación de uso de suelo por medio de imágenes satelitales del Chaco Paraguayo. Existen varias posibilidades para construir sobre este Proyecto Final de Carrera:

\begin{itemize}
    \item Implementación y comprobación de la eficacia de más modelos, como por ejemplo las variantes más complejas de U-Net.
    \item La creación de metodologías más rigurosas para la creación de conjuntos de datos de entrenamiento.
    \item Expansión de los modelos para la generación de predicciones de múltiples etiquetas, en lugar de una sola.
    \item Exploración del uso de arquitecturas CNN menos complejas, reduciendo el costo y tiempo de entrenamiento e inferencia.
    \item Trabajos de campo para la exploración de zonas de paleocauces menos estudiados, detectados por modelos predictivos.
    \item Análisis de datos provenientes de otros proveedores de imágenes satelitales, por ejemplo Landsat, haciendo uso de un conjunto sensores diferentes.
\end{itemize}
