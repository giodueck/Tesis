\section{Discusión de los Resultados}
\todo{discusion de los resultados}

% Conclusion:
% - fcn models not appropriate for this task, farseg shows promise and probably needs better training data
% - unet very good in relation to the quality of the training data
% - since training data is best-effort, results are not perfect or even the best possible with the architecture
% - results useful for a first review of an area
% Strengths:
% - short training time
% - simple architectures
% - simple training data setup
% - good for active/humid palaeochannels
% - decent for colmatado palaeochannels
% Limitations:
% - noisy in urban/cultivated areas
% - very dependent on time of year: rainy seasons and droughts both have very negative impacts
% - tipically a result model good for active pc is bad for colmatado, and vice-versa
% Possible improvements:
% - 192 has nicer results than 128, larger could be even better
% - 60m images were used here, larger images could make 20m or 10m resolution useful as well
% - using professional cards or higher end cosumer hardware will lift some limitations related to memory and video-memory
% - training masks can still be improved, will likely eliminate some of the noisy output and make results more consistent
% - u-nets have variants, some are sure to improve results, e.g. unet++

\subsection{Conclusión}

\subsection{Recomendaciones}

\subsection{Contribución}

\subsection{Trabajos futuros}
