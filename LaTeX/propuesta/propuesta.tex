\documentclass[a4paper, 11pt]{article}
\usepackage[margin=2.2cm]{geometry}

\usepackage[utf8]{inputenc}
\usepackage[spanish]{babel}
\usepackage{csquotes}
\usepackage{hyperref}
\usepackage{graphicx}
\graphicspath{ {../img/} }
\usepackage[backend=biber,style=numeric]{biblatex}
\addbibresource{propuesta.bib}
\usepackage[table,svgnames]{xcolor}
\hfuzz=15pt

% \title{Uso de Redes Neuronales Convolucionales para Interpretación de Imágenes Satelitales}
% \author{Giovanni Dueck\\\UrlFont{giodueck@gmail.com}\\\\{\bf Universidad Católica Nuestra Señora de Asunción}}
% \date{Junio, 2024}

\begin{document}

% \maketitle
\begin{center}
    \Large{\bf Uso de Redes Neuronales Convolucionales para Interpretación de Imágenes Satelitales}

    Propuesta de Tema - Proyecto Final de Carrera Ing. Informática

    {\bf Alumno:} Giovanni Dueck

    {\bf Tutor:} Alberto Ramírez - {\bf Cotutor:} Felix Carvallo
\end{center}

\section*{Introducción y Motivación}

Imágenes satelitales o teledetección remota se refiere a imagenes capturadas por un sensor montado en un satélite artificial, para extraer información. Estas imágenes contienen información multiespectro, es decir, además de la luz visible se toman imágenes de bandas invisibles como por ejemplo la luz infrarroja. \autocite{globalforestlink-how-sat-imaging-work}

Para la captura de estas imágenes se emplean varios métodos, que se dividen en dos categorías: sensores pasivos recolectan radiación electromagnética reflejada del sol, mientras que sensores activos emiten su propia radiación y captan la reflexión de la tierra. Sensores pasivos requieren de una cantidad importante de energía para operar, pero tienen la ventaja de operar a cualquier hora del día y la capacidad de crear imágenes en bandas que el sol no emite. \autocite{globalforestlink-how-sat-imaging-work}

Diferentes suelos, vegetación, o humedad reflejan diferentes bandas de radiación. Estas imágenes son usadas en varias aplicaciones, desde Sistemas de Información Geográfica y mapas a meteorología y monitoreamiento de la salud de vegetación forestal. Un índice bastante común es el Índice de Vegetación de Diferencia Normalizada, o NDVI por sus siglas en inglés, el cual es usado para estimar la cantidad, calidad y desarrollo de la vegetación con base a la mediación. Estos datos ya se usan en sistemas de advertencia temprana de sequías y la predicción del rendimiento de la agricultura en los Estados Unidos a partir de los datos de la NASA. \autocite{earthdata-vegetation}

La importancia de la producción agropecuaria y agroganadera en el Paraguay también invita a considerar estas tecnologías para el monitoreamiento de la salud de la vegetación y el uso adecuado de la tierra. Actualmente, ya es están empleando tecnologías de teledetección y el NDVI en el sector agrícola en aplicaciones como la detección de malezas y predicción del orden idel de cosecha de campos cultivados. \autocite{onesoil-agricultura-paraguay}

Con aproximadamente la mitad del territorio paraguayo hacia el norte del río Paraguay en la región semi-árida del Chaco, tecnologías que alivien las sequias y precipitación baja son muy valiosas, tanto para la agricultura y ganadería en las estancias chaqueñas como para centros poblacionales más aislados como por ejemplo las misiones indígenas. Estos pueblos más aislados generalmente son caracterizados por pobreza, lo que se manifiesta en la falta de alimento y agua potable.

Los paleocauces son formaciones geológicas que han sido propuestas como reservorios o conductos para el flujo subterráneo de agua dulce. Se consideran principalmente paleocauces arenosos, y estos pueden ser aprovechados para acceder al agua en áreas en las que la distribución habitual del agua no existe o está dificultada de alguna forma. \autocite{wikipedia-paleochannel} Con la abundancia de paleocauces en el Chaco central (ocupan un 15\% de la región), esta propuesta es una bastante prometedora que ya ha sido considerada en investigaciones anteriores. \autocite{conacyt-sistemas-captacion-agua}

\section*{Objetivo General}
Creación de modelos a partir de redes neuronales convolucionales para la clasificación y/o caracterización de imágenes satelitales.

\section*{Objetivos Específicos}
\begin{enumerate}
    \item Análisis de series temporales a partir de imágenes satelitales correspondientes a la region occidental del Paraguay
    \item Identificación y clasificación de componentes de uso de suelo
    \item Determinación de áreas de ocurrencia de paleocauces
\end{enumerate}

\section*{Metodología y Cronología de Trabajo}

\begin{enumerate}
    \item Estudio y análisis el estado del arte.
    \item Recopilación de imágenes a ser utilizadas en las series temporales a analizar.
    \item Creación de modelos de identificación y clasificación de uso de suelo.
    \item Identificación de patrones detectados por los modelos creados.
    \item Creación de modelos de identificación de áreas de ocurrencia de paleocauces.
    \item Elaboración del libro de PFC.
\end{enumerate}

\begin{table}[h!]
    \centering
    \small
    \begin{tabular}{ |>{\bf \columncolor{OrangeRed}} c|c|c|c|c|c|c|c|c|c| }
        \hline
        \rowcolor{OrangeRed}
        \bf Etapa & \bf Jul. & \bf Ago. & \bf Sep. & \bf Oct. & \bf Nov. & \bf Dic. & \bf Ene. & \bf Feb. & \bf Mar. \\
        \hline
        1. & \cellcolor{Orange} & \cellcolor{Orange} & \cellcolor{Orange} & \cellcolor{Orange} & & & & & \\
        \hline
        2. & \cellcolor{Orange} & \cellcolor{Orange} & \cellcolor{Orange} & & & & & & \\
        \hline
        3. & & \cellcolor{Orange} & \cellcolor{Orange} & \cellcolor{Orange} & & & & & \\
        \hline
        4. & & & \cellcolor{Orange} & \cellcolor{Orange} & \cellcolor{Orange} & \cellcolor{Orange} & \cellcolor{Orange} & \cellcolor{Orange} & \\
        \hline
        5. & & & & & \cellcolor{Orange} & \cellcolor{Orange} & \cellcolor{Orange} & \cellcolor{Orange} & \\
        \hline
        6. & & & & & & & \cellcolor{Orange} & \cellcolor{Orange} & \cellcolor{Orange} \\
        \hline
    \end{tabular}
\end{table}

\printbibliography

\end{document}
